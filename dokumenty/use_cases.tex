
\documentclass[a4paper]{article}

\usepackage[utf8]{inputenc}
\usepackage{polski}

\usepackage{titlesec}
\usepackage{enumitem}
\usepackage[bookmarks=true]{hyperref}

\hypersetup{
    pdftitle={Przypadki użycia Trend Spotter},
    pdfauthor={Bartłomiej Wrona},
}

\def\version{1.1}

\begin{document}


\titleformat*{\subsubsection}{\normalfont\fontsize{10}{14}\selectfont}

\thispagestyle{empty}
\begin{flushright}\begin{bfseries}
	\Huge Przypadki użycia\\
	\vspace{1.5cm}
	\huge dla systemu\\
	\vspace{1.5cm}
	\Huge Trend Spotter\\
	\vfill
	\LARGE Wersja \version\\
\end{bfseries}\end{flushright}


\newpage

\subsection*{Historia Wersji}

\begin{center}
    \begin{tabular}{|c|c|c|p{6cm}|}
        \hline
	    \textbf{Wersja} & \textbf{Data} & \textbf{Autor / Autorzy} & \multicolumn{1}{|c|}{\textbf{Opis}}\\
        \hline
        0.0 & 12.03.2018 & Marek Smolarczyk & Struktura dokumentu\\
        \hline
	    1.0 & 16.04.2018 & Bartłomiej Wrona & Podstawowe przypadki użycia\\
	    \hline
	    1.1 & 16.04.2018 & Bartłomiej Wrona & Poprawienie przypadków użycia\\
	    \hline
    \end{tabular}
\end{center}
\vspace{1cm}

\subsection*{Zespół}
\begin{enumerate}
	\itemsep-0.1cm
    \item Marek Smolarczyk (386007)
    \item Miron Szewczyk (383504)
    \item Michał Pawłowski (371303)
    \item Bartek Wrona (386389)
\end{enumerate}

\newpage


\section{Logowanie i rejestracja}
\subsection*{Opis}
Użytkownik może logować się za pomocą konta Google, Facebook lub przy użyciu osobnego konta w serwisie Trend Spotter. Logowanie odbywa się za pośrednictwem strony internetowej. 
\subsection*{Aktorzy}
\begin{enumerate}
\item Użytkownik
\item Serwer
\end{enumerate}

\subsection*{Stan początkowy}
Użytkownik jest niezarejestrowany lub niezalogowany.

\subsection{Przebieg}

\subsubsection{Jeśli adres IP użytkownika został umieszczony na liście blokowanych adresów użytkownikowi nie wyświetla (przez czas blokady \ref{login-czas-blokady}) się ikona logowania i rejestracji w pasku nawigacyjnym strony internetowej.} 

\subsubsection{Użytkownik klika w ikonę logowania i rejestracji.}

\subsubsection{Jeśli użytkownik chce się zalogować to przechodzi do punktu \ref{wybor-login}}

\subsubsection{Użytkownikowi zostaje wyświetlony formularz rejestracyjny.}


\subsubsection{Użytkownik wprowadza do formularza email oraz hasło i zatwierdza go klikając guzik.}

\subsubsection{Serwer zapamiętuje informacje z formularza w bazie danych.}

\subsubsection{Użytkownik wybiera jedną z 3 dostępnych metod logowania: za pomocą konta Google \ref{login-google}, konta Facebook \ref{login-fb}, lub konta w naszym serwisie. \label{wybor-login}}

\subsubsection{Użytkownikowi zostaje wyświetlony formularz do logowania.}

\subsubsection{Użytkownik wprowadza do formularza login i hasło i zatwierdza go klikając guzik.}

\subsubsection{Serwer sprawdza czy dane użytkownika są poprawne.}

\subsubsection{Jeśli dane użytkownika są niepoprawne i liczba niepoprawnych prób logowania nie przekracza limitu dozwolonych logowań \ref{limit-logowan}, to zostaje ponownie przekierowany do formularza logującego \ref{wybor-login}}.

\subsubsection{Jeśli liczba niepoprawnych prób została przekroczona, użytkownik zostaje przekierowany na stronę główną, a IP z którego próbował się zalogować zostaje zablokowany na określony czas \ref{login-czas-blokady}}

\subsubsection{Użytkownik zostaje zalogowany i przekierowany na stronę główną. \label{login-ok}}

\subsection{Alternatywny przypadek użycia, logowanie za pomocą konta Facebook. \label{login-fb}}

\subsubsection{Użytkownik zostaje przekierowany na stronę Facebook'a.}

\subsubsection{Jeśli użytkownik już jest zalogowany to następuje skok do \ref{login-fb-zgoda}.}

\subsubsection{Użytkownik wprowadza login i hasło do konta Facebook.}

\subsubsection{Użytkownik zgadza się na przekazanie informacji o adresie email do serwera. Nastepuje skok do \ref{login-ok}. \label{login-fb-zgoda}}


\subsection{Alternatywny przypadek użycia, logowanie za pomocą konta Google. \label{login-google}}

\subsubsection{Użytkownik zostaje przekierowany na stronę Google'a.}

\subsubsection{Jeśli użytkownik jest zalogowany następuje skok do \ref{login-google-zgoda}.}

\subsubsection{Użytkownik wybiera jedno z zapamiętanych kont lub loguje się na konto Google.}

\subsubsection{Użytkownik zgadza się na przekazanie informacji o adresie email do serwera. Nastepuje skok do \ref{login-ok}. \label{login-google-zgoda}}


\subsection{Wymagania}

\subsubsection{Górne ograniczenie liczby dozwolonych błędnych logowań w ciągu godziny wynosi 5. \label{limit-logowan} }

\subsubsection{Czas blokady spowodowanej wielokrotnym nieudanym logowaniem wynosi 15 minut. Po tym czasie liczba błędnych wprowadzeń jest zerowana, a blokada zostaje zdjęta. \label{login-czas-blokady}}

\subsubsection{Logowanie za pomocą konta Facebook / Google jest możliwe tylko w przypadku posiadania konta w danym serwisie.}

\subsection*{Stan końcowy}
Użytkownik jest zarejestrowany i zalogowany.


\newpage


\section{Wyświetlanie historii popularności wpisanych przez użytkownika wyrazów lub fraz
odpowiadających zbiorom wyrazów.}

\subsection*{Opis}
Użytkownikowi zostaje przedstawiony diagram związany z historią popularności wprowadzonych wyrazów. Diagram ma odzwierciedlać względną popularność wprowadzonych wyrazów względem najpopularniejszych wyrazów.

\subsection*{Aktorzy}
\begin{enumerate}
\item Użytkownik

\end{enumerate}

\subsection*{Stan początkowy}
Strona główna. Użytkownik jest zalogowany.

\subsection{Przebieg}
\subsubsection{Użytkownik klika w przycisk popularności (\ref{popularnosc}) w pasku nawigacyjnym.}
\subsubsection{Użytkownik zostaje przekierowany do formularza.}
\subsubsection{Użytkownik wpisuje wyrazy i frazy oddzielając je średnikami, a następnie zatwierdza. \label{popularnosc-formularz}}
\subsubsection{Użytkownikowi zostaje przedstawiony diagram związany z historią popularności wprowadzonych słów, oraz formularz \ref{popularnosc-formularz}}

\subsection{Wymagania}
\subsubsection{Grafika powiązana z przyciskiem ma kojarzyć się użytkownikom z popularnością słów. \label{popularnosc}}
\subsubsection{Diagram i formularz mają znajdować się na tej samej stronie. Użytkownik może wielokrotnie wpisywać swoje zapytania. Zatwierdzenie ma być ostatnią czynnością ze strony użytkownika przed wyświetleniem.}


\subsection*{Stan końcowy}

Strona wyświetlająca historię popularności. Użytkownik nadal jest zalogowany.

\newpage

\section{Wyświetlanie najpopularniejszych zapytań użytkowników.}
\subsection*{Opis}
Użytkownikowi wyświetlany jest ranking najpopularniejszych zapytań. W rankingu zamieszczone są informacje o:
\begin{enumerate}
\item miejscu w rankingu (obliczanym na podstawie liczby zapytań w okresie bazowym \ref{okres-bazowy})
\item liczbie zapytań w ciągu ostatnich 3 okresów bazowych \ref{okres-bazowy}
\item wyszukiwanej frazie
\end{enumerate}

\subsection*{Aktorzy}
\begin{enumerate}
\item Użytkownik
\item Serwer

\end{enumerate}

\subsection*{Stan początkowy}
Strona główna. Użytkownik jest zalogowany.

\subsection{Przebieg}
\subsubsection{Użytkownik klika w przycisk rankingowy w pasku nawigacyjnym. }
\subsubsection{Serwer generuje ranking na podstawie posiadanych danych.}
\subsubsection{Użytkownik zostaje przekierowany na pierwszą stronę rankingu.}
\subsubsection{Użytkownik może dowolną liczbę razy kliknąć w numer innej strony, co powoduje przekierowanie go na inną stronę rankingową.}


\subsection{Wymagania}


\subsubsection{Okres bazowy wynosi 1 miesiąc. \label{okres-bazowy}}
\subsubsection{Obrazek powiązany z przyciskiem rankingowym powinien być powszechnie kojarzony z rankingami.}
\subsubsection{Ranking ma przypominać spaginowaną tabelę.}
\subsubsection{Na każdej stronie rankingowej ma być conajwyżej 30 wpisów.}
\subsubsection{Numery stron (niekoniecznie wszystkich) będące guzikami mają znajdować się pod tabelą.}

\subsection*{Stan końcowy}
Strona wyświetlająca popularność zapytania użytkowników. Użytkownik nadal jest zalogowany.


\newpage

\section{Lista pozostałych przypadków użycia}
\begin{enumerate}
	\item Wyświetlanie wyrazów, fraz szybko zyskujących na popularności (podczas wyliczania ogólnych statystyk pomijane są wyrazy z wcześniej przygotowanej listy, na przykład spójniki).
	\item Wyświetlanie zapytań ograniczonych do poszczególnych serwisów, poprzez wybór jednego lub wielu serwisów z rozwijanej listy.
	\item Wyświetlenia wszystkich lub części komentarzy spełniających jego kryteria wyszukiwania.
    \item Włączenia za pomocą checkboxa ignorowania polskich znaków w danym zapytaniu.
    \item Włączenia za pomocą checkboxa ignorowania wielkich liter w danym zapytaniu.
	\item Dla każdego użytkownika jest przetrzymywana historia jego zapytań. Jest możliwe ponowne wykorzystanie dawnych zapytań poprzez wybranie ich z listy.



\end{enumerate}


\end{document}
