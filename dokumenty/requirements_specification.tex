\documentclass[a4paper]{scrreprt}

\usepackage[utf8]{inputenc}
\usepackage{polski}

\usepackage{enumitem}
\usepackage[bookmarks=true]{hyperref}

\hypersetup{
    bookmarks=false,
    pdftitle={Specyfikacja Wymagań Trend Spotter},
    pdfauthor={Marek Smolarczyk},
}

\def\version{1.0}

\begin{document}

\begin{flushright}\begin{bfseries}
	\Huge Specyfikacja Wymagań\\
	\vspace{1.5cm}
	\huge dla systemu\\
	\vspace{1.5cm}
	\Huge Trend Spotter\\
	\vfill
	\LARGE Wersja \version\\
\end{bfseries}\end{flushright}

\chapter*{Historia Wersji}

\begin{center}
    \begin{tabular}{|c|c|c|l|}
        \hline
	    \textbf{Wersja} & \textbf{Data} & \textbf{Autor / Autorzy} & \multicolumn{1}{|c|}{\textbf{Opis}}\\
        \hline
	    1.0 & 12.03.2018 & Marek Smolarczyk & Struktura dokumentu i podstawowe wymagania\\
        \hline
	     & & & \\
        \hline
    \end{tabular}
\end{center}

\newpage

\section*{Opis ogólny}
System ma umożliwiać wykrywanie kształtujących się trendów i analizę historii popularności różnych tematów. Z systemu będą mogli korzystać zarówno klienci indywidualni jak i firmy i organizacje.

Na system składają się serwer WWW oraz crawler.

\section*{Założenia i zależności}
System może gromadzić anonimowe informacje o sposobie użytkowania aplikacji.

System nie udostępnia funkcji rejestracji w usłudze.

W dalszej części dokumentu zakładamy, że użytkownik korzysta z aplikacji na jednej z podanych niżej przeglądarek:
\begin{enumerate}[label=\textbullet]
	\itemsep-0.1cm
	\item Firefox (min. wersja $\cdot$)
	\item Chrome $\cdots$
\end{enumerate}

\section*{Wymagania funkcjonalne}
Wymagania graficzne aplikacji (wraz z przykładami) opisane są w pliku \href{}{$\cdots$}.

Aplikacja będzie umożliwiała korzystanie z przetworzonych danych zgromadzonych przez crawler'a w następujące sposoby:
\begin{enumerate}
	\itemsep-0.1cm
	\item wyświetlanie historii popularności wyszukanych wyrazów, fraz odpowiadających wyrażeniom regularnym;
	\item wyświetlanie najpopularniejszych zapytań użytkowników;
	\item wyświetlanie wyrazów, fraz szybko zyskujących na popularności;
	\item ograniczanie zapytań do poszczególnych serwisów.
\end{enumerate}

\section*{Wymagania niefunkcjonalne}
\begin{enumerate}
	\itemsep-0.1cm
	\item Crawler będzie obsługiwał maksymalnie 10 serwisów. Dane zebrane przez crawler'a będą przechowywane w bazie danych nie dłużej niż 6 miesięcy.

	\item Dane przechowywane przez system  będą związane z użytkownikami, tak więc będzie koniecznie szyfrowane.

	\item System zapewnia obsługę pojedynczego zapytania użytkownika średnio w nie więcej niż 5 sekund.

	\item System będzie umożliwiał jednoczesne korzystanie z usługi przez nie więcej niż 10 użytkowników.

	\item System zapewnia średni czas pomiędzy awariami nie krótszy niż 2000 godzin oraz średni czas naprawy około 12 godzin.

\end{document}

-odporność na ataki sql-injection ze strony zarówno zapytań jak i zebranych komentarzy F
-logowanie, przetrzymywanie i wyświetlanie histori zapytań, możliwość uruchomienia zapytania z poziomu histori zapytań. Logowanie za pomocą gmaila i facebooka F

-Ograniczenie serwisów poprzez wybór z listy, można wybrać wiele F
-Crawler crawluje każdą domenę niezależnie. Administrator ma opcję wymusznie na crawlerze przwejrzenia całej domeny, operacja ta może zając wiele dni. NF
-Crawler nie powinien wchodzić na tę samą domenę częściej niż co 10s. NF
-Działenie crawlera opiera się na powracanie na wybrane strony z częstotliwością zależącą od dat zebrancyh na danej stronie komentarzy. Crawler częściej odwiedza strony nowsze. NF
-Crawler z każdej strony zbiera komentarze i daty ich utworzenia poczym przesyła je do bazy danych. NF
-Baza danych przetrzymuje pary komentarz, data utworzenia, link do strony gdzie został znaleziony. NF
-Baza danych przetrzymuje tylko pierwsze 8 tysięcy znaków każdego komentarza. NF
-Komentarze są przetrzymywane przy pomocy kodowania UTF8 NF
-Przy wyszukiwaniu jest checkbox czy rozróżniać polskie znaki i wielkie i małe litery F
-Średni czas upływający od storzenia komentarza do jego zaindeksowania powinien być nie większy niż 24h NF
-Będzie utworzona lista słów mało znaczących słów, pomijanych przy wyświetlaniu statystyk F
-Zamiast regexa wyszukujemy pojedynczych lub kilku słów w bazie danych F

